%based on imperial example provided + hra example at http://www.hra-decisiontools.org.uk/consent/examples.html 
\documentclass[a4paper,10pt]{article}

%set font to Arial
%\usepackage{fontspec}
%\setmainfont{Arial}
\usepackage{helvet}
\renewcommand{\familydefault}{\sfdefault}

%graphics
\usepackage{graphicx}
%\usepackage{subfigure}
\usepackage{pslatex}
\usepackage{pstricks}

%math equations
\usepackage{amsmath}

%python code
%\usepackage{minted}

%headers
\usepackage{lastpage}
\usepackage{fancyhdr}
\pagestyle{fancy} 
\lhead{Rhif adnabod y prosiect: 203355}
\chead{}
\rhead{clinicaltrials.gov reference}
\lfoot{IPF JES}
\cfoot{Taflen Wybodaeth i'r sawl sy'n cymryd rhan f0.4 \hspace{1cm} \today}
\rfoot{\thepage\ of \pageref{LastPage}}


%display URLS
\usepackage{url}

%hyperlinks
\usepackage{hyperref}

%comments
\usepackage{verbatim} 

%nice tables
\usepackage{booktabs}
\newcommand{\ra}[1]{\renewcommand{\arraystretch}{#1}}

%multi rows for the nice tables
%\usepackage{multirow} 

%nice diplay of code
%\usepackage{minted}

%nice references
\usepackage[super]{natbib}

%some maths
\usepackage{amsmath}

%margins
%\usepackage{geometry}
%\geometry{verbose,a4paper,tmargin=60mm,bmargin=25mm,lmargin=25mm,rmargin=25mm}

%in line citations
%\usepackage{bibentry}

%\hyphenpenalty=10000

%\nobibliography*

\newpage\title{\bf Taflen Wybodaeth i'r sawl sy'n cymryd rhan} 
\date{}
%\author{Carl Reynolds \\
%\small National Heart \& Lung Institute, Imperial College London }

\begin{document}
\pagestyle{fancy} 


\pagenumbering{arabic}

%\pagestyle{empty}

%\maketitle

\section*{Taflen wybodaeth i'r sawl sy'n cymryd rhan}

\subsection*{Astudiaeth ar y cysylltiad rhwng swyddi � Ffibrosis Idiopathig yr Ysgyfaint (Astudiaeth IPF)}
\subsection*{Astudiaeth ymchwil sy'n ceisio darganfod a yw llefydd gwaith yn achosi ffibrosis idiopathig yr ysgyfaint (IPF)}

Y prif ymchwilydd yw Dr Carl Reynolds, cymrawd ymchwil clinigol yng Ngholeg Imperial, Llundain. 

\section*{RHAN 1}
\subsection*{Allwch chi helpu efo astudiaeth ymchwil?}

\begin{itemize}
 \item Hoffwn eich gwahodd i gymryd rhan mewn astudiaeth ymchwil. Cyn i chi benderfynu, rydym ni'n awyddus i chi ddeall pam bod yr ymchwil yn cael ei wneud, a beth fyddai'n ei olygu i chi.  
 \item Bydd aelod o'n t�m yn mynd drwy'r daflen wybodaeth efo chi, ac yn ateb unrhyw gwestiynau sydd gennych chi. Dylai hyn gymryd ryw 10 i 15 munud.
 \item Mae croeso i chi siarad am yr astudiaeth efo eraill, a gofyn i ni os nad ydych chi'n deall unrhyw beth.
\end{itemize}

\subsection*{Beth yw pwrpas yr astudiaeth?}
\begin{itemize}
 \item Cyflwr yw ffibrosis idiopathig  yr ysgyfaint (IPF) sy'n achosi creithio ar yr ysgyfaint. Mae'r creithio yn difrodi'r codenni aer sy'n gadael i ocsigen symud i'r gwaed ac i organau hanfodol. Mae'n gyflwr difrifol sy'n achosi peswch, diffyg anadl a blinder. 
 \item Ni wyddom beth sy'n achosi'r cyflwr ond mae'n dod yn fwy cyffredin yn Lloegr, yr Alban a Chymru ble mae'n effeithio ar fwy na 4000 o bobl pob blwyddyn. Pobl sy'n h?n na 40 sy'n cael y cyflwr fel arfer; mae'n fwy cyffredin mewn dynion ac mewn rhannau o'r wlad ble mae hanes o ddiwydiant trwm. 
 \item Bydd yr astudiaeth hon yn helpu i weld faint y mae'r amgylchedd gwaith yn Lloegr, yr Alban a Chymru i'w gyfrif am y cyflwr. Bydd hyn yn ein helpu ni i ddeall achosion y cyflwr, gofalu bod pobl yn cael y driniaeth gywir a'r iawndal y mae ganddynt yr hawl i'w gael, a gofalu bod y mesurau yn y gweithle yn iawn fel ein bod ni'n amddiffyn gweithwyr ac yn atal y cyflwr yn y dyfodol. 
\end{itemize}


\includegraphics[width=2.5cm]{fig/wellcome-logo-black.jpg}
\hspace{1cm}
\includegraphics[width=4cm]{fig/lungsatwork-logo.jpg}
\hspace{1cm}
\includegraphics[width=4cm]{fig/imperial-logo.jpg}

\subsection*{Pam fy mod i wedi cael fy newis?}
\begin{itemize}
 \item Mae'r astudiaeth yn gweithio drwy gymharu pobl sydd �'r cyflwr (gr?p achosion) gyda phobl heb y cyflwr (gr?p rheoli). Mae'n rhaid cael y ddau gr?p yn yr astudiaeth.
 \item Rydych chi wedi cael eich dewis i gymryd rhan yn yr astudiaeth fel \textbf{achos} os ydych chi wedi cael diagnosis newydd o'r cyflwr\@.
 \item Rydych chi wedi cael eich dewis i fod yn rhan o'r \textbf{gr?p rheoli} yn yr astudiaeth os nad oes gennych chi IPF ond eich bod chi wedi cael apwyntiad ysbyty fel claf allanol yn ddiweddar, a'ch bod chi tua'r un oed � chleifion sydd newydd gael diagnosis o'r cyflwr\@.  
\end{itemize}

\subsection*{Oes raid i mi gymryd rhan?}
\begin{itemize}
 \item Chi sy'n penderfynu os ydych chi am gymryd rhan yn yr ymchwil neu beidio. Fe wnawn ni ddisgrifio'r astudiaeth a mynd drwy'r llyfryn gwybodaeth hwn efo chi. 
 \item Os byddwch chi'n cytuno i gymryd rhan, fe wnawn ni ofyn i chi ddarllen ffurflen gydsynio a'i harwyddo. 
 \item Cewch dynnu yn �l unrhyw bryd, heb roi rheswm. Ni fydd hyn yn effeithio ar unrhyw ofal a gewch chi. 
\end{itemize}

 
\subsection*{Pwy yw'r ymchwilwyr?}
T�m ymchwil o Goleg Imperial Llundain, Ysbytai GIG Gofal Iechyd Coleg Imperial, ac Ysbytai Gofal Sylfaenol GIG Sheffield a fydd yn cynnal yr ymchwil. Ariennir yr ymchwil gan y Wellcome Trust. Y prif ymchwilwyr yw:
\begin{itemize}
    \item Dr Carl Reynolds, Cymrawd Hyfforddiant Ymchwil Clinigol Wellcome Trust, NHLI (Coleg Imperial Llundain). (Prif Ymchwilydd)
 \item Yr Athro Paul Cullinan, Athro, meddyg ymgynghorol anrhydeddus (meddyginiaeth anadlu). Meddyginiaeth Galwedigaethol ac Amgylcheddol, NHLI (Coleg Imperial Llundain), Ysbyty Royal Brompton, Llundain. Cyd-benodiad; � daliadaeth. (Cyd-ymchwilydd)
 \item Dr Chris Barber, Meddyg ymgynghorol (meddyginiaeth anadlu), Ysbyty Northern General, Sheffield. (Cyd-ymchwilydd)
 \item Dr Sara De Matteis, Darlithydd Clinigol, NHLI (Coleg Imperial Llundain). (Cyd-ymchwilydd)
\end{itemize}

\section*{RHAN 2}

\subsection*{Beth fydd yn digwydd i chi os wnewch chi gymryd rhan?}
\begin{itemize}
 \item Os wnewch chi gytuno i gymryd rhan, bydd yr ymchwilydd yn cysylltu � chi i drefnu cyfweliad dros y ff�n ar adeg cyfleus i chi. 
 \item Ni fydd y cyfweliad dros y ff�n yn para mwy nag awr.
 \item  Yn ystod y cyfweliad, fe gewch eich holi am\begin{itemize}
                                                                \item Bob swydd rydych chi wedi'i gwneud ers gadael yr ysgol; efallai y byddwn ni'n gofyn i chi am swyddi'r bobl rydych chi wedi byw efo nhw hefyd
                                                                \item Ble rydych chi wedi byw  
                                                                \item Eich arferion ysmygu yn ystod eich oes 
                                                               \end{itemize}
\item Cysylltir � chi i drefnu prawf gwaed i edrych ar eich tueddiad genetig o ran IPF\@. Os oes modd, fe gewch chi'r prawf y tro nesaf i chi gael profion gwaed, i osgoi prawf ychwanegol. Os na fydd hyn yn bosib, fe'i drefnir ar adeg a rhywle cyfleus i chi. Fe dalwn ni am unrhyw gostau teithio rhesymol a gewch chi o fod yn rhan o'r astudiaeth - cytunir ar y rhain o flaen llaw. 
\item Gyda'ch caniat�d chi, fe wnawn ni ysgrifennu at eich meddyg teulu i ddweud eich bod chi'n cymryd rhan. 
\item Fe wnawn ni ddweud y canlyniadau wrthych chi. Efallai na welwn ni unrhyw beth a fydd o ddefnydd i chi. Os gwelwn ni unrhyw beth, fe ddywedwn wrth eich t�m clinigol, gyda'ch caniat�d chi.
\end{itemize}

\subsection*{Pam ydych chi'n gofyn am brawf gwaed?}

Rydym ni eisiau gwybod a yw amgylcheddau gwaith yn achosi ffibrosis idiopathig yr ysgyfaint. Ar gyfer y rhan fwyaf o glefydau, bydd rhywun yn cael y cyflwr yn dibynnu ar yr hyn y maent yn dod i gysylltiad ag ef yn eu hamgylchedd, a'r DNA neu'r genynnau y c�nt eu geni efo nhw.    

Mae IPF yn gyflwr prin. Nid yw'n gyflwr sy'n rhedeg yn y teulu fel arfer, ond mae'n fwy cyffredin mewn pobl sydd � gwahaniaethau genetig neilltuol, fel newid bach sy'n effeithio ar y mwcws  yn ein llwybrau anadlu (o'r enw MUC5B rs35705950). Mae'r prawf gwaed yn ein helpu ni i weld ai'r amgylcheddau gwaith, ar ben y gwahaniaethau genetig hyn, sy'n achosi IPF.   

\subsection*{Beth fydd canlyniad y prawf gwaed yn ei olygu i mi?}

Os gwelwn ni eich bod yn cario'r gwahaniaeth genetig MUC5B rs3570950, nid  yw'n golygu bod y cyflwr arnoch chi neu y bydd aelodau eich teulu yn cael y cyflwr.  

Yn �l astudiaethau, rydych chi ryw chwe gwaith yn fwy tebygol o gael IPF os ydych chi'n cario MUC5B rs3570950. Er hynny, mae IPF yn brin (llai nag un mewn 2000 o bobl yn y DU sy'n cael diagnosis o'r cyflwr ar ryw adeg yn eu bywyd), felly mae'r risg o IPF ar y cyfan i bobl sy'n cario MUC5B rs3570950 yn dal yn isel iawn. 

\subsection*{A oes manteision i gymryd rhan?}

Mae'n annhebygol y bydd yr astudiaeth yn eich  helpu chi'n  bersonol. Efallai y bydd yr wybodaeth a gawn ni o'r ymchwil yn helpu i ddeall achosion IPF, gofalu bod pobl yn cael y driniaeth gywir a'r iawndal y mae ganddynt yr hawl i'w gael, a gofalu bod y mesurau ar gemegau yn y gwaith yn iawn fel ein bod ni'n amddiffyn gweithwyr ac yn atal y cyflwr yn y dyfodol.

Efallai y bydd cleifion sydd � chyflyrau a achosir gan eu gwaith yn cael iawndal. Ar hyn o bryd, nid yw cleifion sydd ag IPF  yn debygol o gael iawndal oherwydd nid ydym yn gwybod mai gwaith sy'n ei achosi. Os gwelwn ni bod amgylcheddau gwaith  yn achosi IPF i rai pobl, yna gall hyn newid i gleifion yn y dyfodol.   
ddiad
\subsection*{A oes unrhyw risgiau i gymryd rhan?}

Y risg fwyaf i chi wrth gymryd rhan yn yr astudiaeth hon yw'r risg i'ch gwybodaeth breifat gael ei datgelu. Er mwyn lleihau'r risg o golli cyfrinachedd, ni chaiff eich ymateb yn y cyfweliad (na'r sampl gwaed) eu labelu gyda gwybodaeth breifat amdanoch chi. Bydd gwybodaeth am eich cyfweliad yn cael ei chadw wedi'i hamgryptio ar gyfrifiadur mewn swyddfa dan glo. Caiff samplau gwaed eu storio rhywle diogel. Ni chewch eich enwi mewn unrhyw adroddiad nac ychwaith drwy gyhoeddi'r astudiaeth hon na'i chanlyniadau. 

Mae risg y gwnawn ni ddod ar draws rhywbeth sy'n bwysig i'ch iechyd. Gallai hyn beri gofid i chi. Os ddown ni ar draws unrhyw beth y credwn allai fod yn bwysig i'ch iechyd chi, fe ddywedwn wrthych chi, a gyda'ch caniat�d chi, fe ddywedwn wrth eich meddyg teulu a'r meddygon yn yr ysbyty. 

Adolygwyd yr astudiaeth gan Bwyllgor Moeseg Ymchwil Nottingham 1.

\subsection*{Beth fydd yn digwydd ar �l i'r ymchwil orffen?}

Bydd crynodeb o'r canlyniadau ar gael ac fe gewch chi gopi dim ond i chi ofyn. Bydd data o'r astudiaeth, gan gynnwys data di-enw heb ei  brosesu, ar gael i'r gymuned academaidd ehangach, llunwyr polisi, drwy ei gyhoeddi a'i gyflwyno mewn cyfarfodydd cenedlaethol a rhyngwladol ar anadlu ac  epidemioleg. Bydd data crynodeb hefyd yn cael ei rannu gyda'r timau gofal sy'n cymryd rhan yn yr astudiaeth.

\subsection*{Beth os oes problem?}

Os oes gennych chi bryder am unrhyw agwedd ar yr astudiaeth, dylech ofyn am gael siarad efo'r ymchwilwyr. Fe wn�nt eu gorau glas i ateb eich cwestiynau. Mae eu manylion cyswllt ar dudalen olaf y llyfryn. Os byddwch chi'n dal yn anfodlon ac eisiau cwyno'n ffurfiol, gallwch wneud hyn drwy gysylltu �'r Patient Advice and Liaison Service (PALS). \\

\newpage

\begin{flushleft}

Patient Advice and Liaison Service (PALS) \\    
Ground floor of the Queen Elizabeth the Queen Mother (QEQM) building, \\
St Mary’s Hospital, \\
South Wharf Road,\\
London W2 1NY.\\
Tel: 020 3312 7777\\
Email: pals@imperial.nhs.uk\\

\end{flushleft}

Mae gan Goleg Imperial Llundain bolis�au yswiriant sy'n berthnasol i'r astudiaeth hon. Os gewch chi niwed neu anaf difrifol o ganlyniad i gymryd rhan yn yr astudiaeth hon, gall fod hawl gennych chi i hawlio iawndal heb orfod profi mai'r Coleg sydd ar fai. Nid yw hyn yn effeithio ar eich hawliau cyfreithiol i hawlio iawndal. 

Os gewch chi niwed oherwydd esgeulustod rhywun, yna efallai bod gennych chi sail ar gyfer  gweithredu cyfreithiol. Os ydych chi eisiau cwyno, neu os oes gennych chi unrhyw bryderon am unrhyw agwedd ar y ffordd y cawsoch eich trin yn ystod yr astudiaeth hon, yna dylech ddweud wth yr Ymchwilydd ar unwaith (Carl Reynolds, manylion cyswllt isod). Mae trefn arferol y Gwasanaeth Iechyd Gwladol  hefyd ar gael i chi. Os byddwch chi'n dal yn anfodlon gyda'r ymateb, gallwch gysylltu � Swyddog Cydymffurfiaeth Ymchwil Rhaglen Cydweithredu Academaidd ym Maes Gwyddorau Iechyd y Coleg. 



\subsection*{Beth fydd yn digwydd i'r wybodaeth a gasglwn?}

Y Prif Ymchwilydd (Dr Carl Reynolds) a fydd yn gyfrifol am sicrhau bod yr holl wybodaeth a gasglwn amdanoch chi yn ystod yr astudiaeth yn cael ei chadw'n gwbl gyfrinachol. Er mwyn i ni gysylltu � chi, bydd rhaid i'ch t�m gofal yn yr ysbyty rannu eich manylion cyswllt efo ni. 
Tynnir eich enw a'ch cyfeiriad oddi ar unrhyw wybodaeth feddygol amdanoch chi a fydd yn gadael yr ysbyty/feddygfa, fel na ellir eich adnabod chi o'r wybodaeth honno.

Bydd yr holl brosesau a ddefnyddir i drin, prosesu, storio a difrodi gwybodaeth amdanoch chi yn cydymffurfio � Deddf Diogelu Data 1998. Bydd yr holl wybodaeth a gasglwn yn cael ei hamgryptio a'i storio ar gyfrifiadur mewn adeilad diogel. Bydd cyfrinair yn amddiffyn y cyfrifiadur hwnnw. Caiff samplau gwaed eu dadansoddi a'u cadw mewn labordy diogel yn y Coleg.  

Caiff samplau a data eu cadw am ddeng mlynedd ar �l i'r astudiaeth orffen. Dim ond aelodau o'r t�m ymchwil a gaiff weld yr wybodaeth a gesglir ac a fydd yn gallu ei chysylltu � chi. Efallai y bydd samplau a data dienw yn cael eu rhannu efo unedau academaidd ac unrhyw gydweithredwyr fferyllol. 
 
\begin{centering}
\subsection*{Diolch am eich diddordeb}
\subsection*{Mae croeso i chi ofyn cwestiynau}
\end{centering}

\vspace{1cm}

\paragraph{Cyswllt} 
\begin{flushleft}
Dr Carl Reynolds / carl.reynolds@imperial.ac.uk / 07737 904 807 \\ 
National Heart and Lung Institute\\
Room G39 Emmanual Kaye Building\\
1b Manresa Road, London, SW3 6LR
\end{flushleft}


 
\end{document}
