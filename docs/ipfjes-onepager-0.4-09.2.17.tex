\documentclass[a4paper,10pt]{article}

%set font to Arial
%\usepackage{fontspec}
%\setmainfont{Arial}
\usepackage{helvet}
\renewcommand{\familydefault}{\sfdefault}

%graphics
\usepackage{graphicx}
%\usepackage{subfigure}
\usepackage{pslatex}
\usepackage{pstricks}

%math equations
\usepackage{amsmath}

%python code
%\usepackage{minted}

%headers
\usepackage{fancyhdr}
\pagestyle{fancy} 
\lhead{IRAS Project ID: 203355}
\chead{}
\rhead{clinicaltrials.gov: NCT03211507 }
\lfoot{IPF JES}
\cfoot{One pager for healthcare professionals v0.4}
\rfoot{9th February 2017}

%display URLS
\usepackage{url}

%hyperlinks
\usepackage{hyperref}

%comments
\usepackage{verbatim} 

%nice tables
\usepackage{booktabs}
\newcommand{\ra}[1]{\renewcommand{\arraystretch}{#1}}

%multi rows for the nice tables
%\usepackage{multirow} 

%nice diplay of code
%\usepackage{minted}

%nice references
\usepackage[super]{natbib}

%some maths
\usepackage{amsmath}

%margins
%\usepackage{geometry}
%\geometry{verbose,a4paper,tmargin=25mm,bmargin=25mm,lmargin=40mm,rmargin=25mm}

%in line citations
%\usepackage{bibentry}

%\hyphenpenalty=10000

%\nobibliography*

\newpage\title{\bf IPF Job Exposures Study (IPF JES)} 
\author{Carl Reynolds \\
\small National Heart \& Lung Institute, Imperial College London }

\pagenumbering{gobble}

\begin{document}


\section*{\centering IPF Job Exposures Study (IPF JES)}

Previous studies have found associations between occupational metal, stone, and wood dust exposures and IPF but have not looked specifically at quantitative asbestos exposure. \\ \\ The question of whether job exposures such as asbestos exposure are important in causing a proportion of cases of IPF arises because:
\begin{itemize}
 \item classical ‘asbestosis’ looks very like IPF
 \item the trends of IPF and asbestos use in the UK are closely aligned; while this does not prove causation it is consistent with a link
 \item it would explain, at least in part, why the disease is more common in men from certain parts of the country
 \item men who have worked with ‘wood’ or ‘metals’ would commonly be exposed also to asbestos
 \item \textit{preliminary} analysis of occupational data for cases and controls obtained from a recent IPF study shows that the odds ratio associated with ever having had a job where asbestos exposure is likely (using a definition from a large mesothelioma case-control study) is 2.8 (95\% CI: 1.42-5.75, p = 0.001)
\end{itemize}

Knowing whether there is a link between job exposures such as asbestos and some cases of IPF would help to better understand the causes of IPF; would change approaches to its current treatment; would have important implications for compensation; and would help to prevent the disease in parts of the world where asbestos is still used widely. \\ \\ We will be recruiting male patients with a new IPF diagnosis (consistent with 2011 ATS/ERS criteria) made between 1/02/2017 and 1/10/2019. 

\paragraph{Study details}
This study will recruit men with new diagnoses of IPF (‘cases’) from a network of UK hospitals. For purposes of comparison a group of men of the same age attending the same hospitals at about the same time for other conditions (‘controls’) will be recruited, in a ratio of 1:1; the total number of participants will be 920. 

Cases and controls will be invited to give details, through a telephone interview, of all the jobs they have had since leaving school. These jobs will be ‘scored’ for the likelihood of their incurring exposure to asbestos; the techniques for doing this are well established. The proportions of so-exposed jobs will be compared between the cases and the controls to investigate whether there is a dose-response relationship for occupational asbestos exposure and IPF. 

Participants will also be invited to provide a blood sample to investigate whether asbestos exposure modifies the association between idiopathic pulmonary fibrosis and a MUC5B promoter (rs35705950) polymorphism which is known to confer susceptibility to IPF.

\paragraph{Contact} Dr Carl Reynolds / carl.reynolds@imperial.ac.uk / 07737 904 807 
\\ National Heart and Lung Institute, Room G39 Emmanual Kaye Building, 1b Manresa Road, London, SW3 6LR.
 
\end{document}

