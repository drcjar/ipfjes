\documentclass[imperial,letterpaper,pagesize,UScommercial9]{scrlttr2}
\usepackage{enumitem}

\begin{document}

\begin{letter}{    
    Proportionate Review Sub-Committee \\        
    East Midlands - Nottingham 1 Research Ethics Committee \\
    The Old Chapel \\
    Royal Standard Place \\
    Nottingham \\
    NG1 6FS} 


\setkomavar{subject}{RE: Idiopathic Pulmonary Fibrosis Job Exposures Study (IPF
JES), Rec reference 17/EM/0021, IRAS project ID 203355}

\opening{Dear Mrs Sarah Lennon,}

Thank-you for your provisional opinion letter dated January 12th 2017. I will respond to your points in turn.

    \begin{enumerate}[]
        \item We have added a section ``Why are you requesting a blood test?" and a section ``What will the result of the blood test mean for me?" to provide information regarding MUC5B rs35705950 SNP and the explain the implications of the test to the participant information sheet. The relevant reference for this content is (Seibold, Max A., et al. ``A common MUC5B promoter polymorphism and pulmonary fibrosis." New England Journal of Medicine 364.16 (2011): 1503-1512.)
        \item We have added a definition of the abbreviation `IPF' to the Letter of Approach.
        \item We have added details of how the researcher (for the Letter of Approach this is the Principle Investigator for the site who will also be a member of the participant's direct care team) got the contact details of the participant
        \item We have made the following changes to the participant information sheet:
        \begin{enumerate}[label=(\alph*)]
            \item Page numbers added to bottom right of patient information sheet
            \item Sentence ``The study has been reviewed by the Nottingham 1 Research Ethics Committee" added to risks and benefits section of patient information sheet
            \item Added ``The lead researcher is Dr Carl Reynolds, clinical research fellow at Imperial College London"      
            \item Contact details are provided for the lead researcher, who is based in London, on the last page
            \item Revised to ``IPF JES is a research study that aims to discover if workplaces are a cause of idiopathic pulmonary fibrosis (IPF)"
            \item Information provided above in 1.
            \item Added a sentence on how compensation would come about:
                ``Diseases that are discovered to be caused by work might get compensation. Currently, patients with IPF are unlikely to get compensation because it is not known to be caused by work. If we find that workplace environments do cause IPF for some people then this may change for patients in the future."
            \item This would not be appropriate given the above. It is possible that in the course of our study a patient may be discovered to have a hitherto undetected significant occupational exposure that leads to their diagnosis of IPF being changed. Study findings will be shared with the patients clinical care team who will remain responsible for diagnosis and treatment of the patient.
            \item Patient Liaison and Advisory Services (PALS) contact details are included.
        \end{enumerate}
        \item Yes/no initial box added to consent form (in keeping with existing form design)
        \item Yes/no initial box added to consent form (in keeping with existing form design)
        \item We have added ``with my permission'' to the wording for participant consent to research team contacting the clinical team regarding clinically significant findings arising from the research on the consent form.
        \item The participant will be informed of study results.
        \item There are two validated historic asbestos exposure assessment components to our interview schedule described below. We have also added more information to the interview schedule to explain that the structured subjective assessment is to be carried out by a trained assessor, that it will include tailored questions to identify substances handled, how handled, and any control measures, and that it will be triggered by the top 15 highest PMR for mesothelioma jobs.
            \begin{enumerate}
                \item Firstly, the use of lifetime occupational histories, coded to SOC90[1] is validated for assessment of asbestos exposure.[1] Secondly, we use John Cherries validated Structured Subjective Assessment of Past Concentrations.[2] (1. Gilham, Clare, et al. ``Pleural mesothelioma and lung cancer risks in relation to occupational history and asbestos lung burden." Occupational and environmental medicine
                    (2015): oemed-2015. 2. Cherrie, John W., and Thomas Schneider. ``Validation of a new method for structured subjective assessment of past concentrations." Annals of Occupational Hygiene 43.4 (1999): 235-245.)
                \item Interviewers will be trained to carry out the structured subjective assessment by John Cherrie. The exact questions asked will vary depending on the occupations reported by the participant. We will use as a trigger for beginning structured subjective assessments jobs with the highest PMR for mesothelioma (the top 15).
            \end{enumerate}
        \item As above
        \item As above (the studies cited are included in the protocol)
        \item We have ensured this - mention of parents in the participant information sheet has been removed
        \item We would prefer not to do this to avoid introducing bias into the study. There is evidence that self-reported asbestos exposure is inaccurate. Many people who have been exposed are unaware, and many people who have no or only trivial exposure report being exposed. People do generally report the jobs that they have had accurately. In the mesothelioma literature asking about jobs and then estimating exposure using proportional mortality ratios has been validated as an accurate means
            for estimating exposure by checking asbestos fibre body counts in lung tissue. We plan to use this method. We would also be worried about introducing recall bias by explicitly asking about asbestos. Because the diagnosis of IPF requires asking about potential occupational exposures it is likely that most IPF patients will have already been asked about asbestos exposure and had more time to think about it. Hence we would expect them to be more likely to recall being exposed to asbestos (when directly asked) than controls.
    \end{enumerate}


\closing{Yours sincerely}

    \includegraphics[width=4cm]{/home/drcjar/Documents/CV/CarlReynoldsSignature.png}

\end{letter}
\end{document}
